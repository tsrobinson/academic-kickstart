\documentclass[11pt, a4paper]{article}
\usepackage[utf8]{inputenc}
\usepackage[left=2.5cm, right=2.5cm, top=2.5cm]{geometry}
\usepackage{changepage}
\usepackage{hyperref}

\setlength{\parindent}{0em}

\begin{document}
 \begin{center}
    \textsc{\huge Thomas Robinson} \\
    Wadham College, University of Oxford, OX1 3PN \\
    +44 (0)7908 231 892 $|$ thomas.robinson@politics.ox.ac.uk $|$ \href{https://ts-robinson.com}{ts-robinson.com}
 \end{center}

% \section*{Academic Employment}
% \begin{tabular}{ll}
%     August 2020 & \textbf{Assistant Professor in Quantitative Comparative Politics} \\
%     & School of Government and International Affairs \\
%     & University of Durham, UK \\

 % \end{tabular}

 \section*{Education}
 \begin{tabular}{ll}
    \textit{Present} & \textbf{University of Oxford} $|$ DPhil in Politics \\
    2018  & \textbf{University of Oxford} $|$ MPhil in Politics (Comparative Government) \\
    2016 & \textbf{University of Oxford} $|$ BA in Philosophy, Politics and Economics
 \end{tabular}

 \section*{Research}

 \subsection*{Interests}
 Ideology, representation and democracy $|$ Direct democratic policymaking $|$ Experimental and computational methodology.

 \subsection*{Peer-Reviewed Articles}

 \begin{enumerate}

 \item \href{https://www.cambridge.org/core/journals/political-analysis/article/multimodes-for-detecting-experimental-measurement-error/37514FC46CF29C7B345DB9881E252150/share/7b059037b0da9182a33316d7f87b2de81b619592}{``Multi-modes for Detecting Experimental Measurement Error."} 2020. With Raymond Duch, Denise Laroze, and Pablo Beramendi. \textit{Political Analysis}, 28(2): 263-283.

 \item \href{https://doi.org/10.1111/ssqu.12584}{``Where Will the British Go? And Why?"}. 2019. With Raymond Duch, Denise Laroze, and Constantin Reinprecht. \textit{Social Science Quarterly}, 100(2): 480-493. 

 \item \href{http://raymondduch.com/files/Populism__Magnet_or_Deterrent_May2019.pdf}{``Populism: Magnet or Deterrent?"} With Raymond Duch, Raymond, Denise Laroze, and Constantin Reinprecht. (\textit{Conditional Acceptance, PSRM}).

 % \begin{adjustwidth}{0.5cm}{0.5cm}
% \textit{We implement conjoint experiments in the UK, India, Chile and China to test the effect of nativist policy prescriptions on emigration preferences. We show that whether these policies matter is highly contingent on the subject's country of origin.}
% \end{adjustwidth}

\end{enumerate}

\subsection*{Under Review}

\begin{enumerate}

\item \href{https://doi.org/10.33774/apsa-2020-3tk40-v3}{``Applying the MIDAS Touch: How to Handle Missing Values in Large and Complex Data."} With Ranjit Lall.
% \begin{adjustwidth}{0.5cm}{0.5cm}
%  \textit{We present a new method of multiple imputation using denoising autoencoder neural networks with dropout. MIDAS is highly accurate, scales to long and wide datasets, and can learn dependencies in the data without prior specification.} \\
%  \end{adjustwidth}

\item \href{https://ts-robinson.com/files/MIDAS_technical.pdf}{``\textbf{midas}: Efficient Multiple Imputation for Large and Complex Data."} With Ranjit Lall and Alex Stenlake.
% \begin{adjustwidth}{0.5cm}{0.5cm}
%  \textit{This paper introduces \textbf{midas}, a Python class for multiply imputing missing values based on neural network methods that is particularly well suited to large and complex data.} \\
%  \end{adjustwidth}

\item ``How Campaigns Respond to Ballot Design? Evidence from Ballot Order Lotteries in Colombia.'' With Nelson Ruiz and Saad Gulzar.
% \begin{adjustwidth}{0.5cm}{0.5cm}
%  \textit{We provide evidence for a new unexplored mechanism for ballot order effects, in which campaigns themselves adjust their behavior based on their randomly assigned ballot position.} \\
% \end{adjustwidth} 


\end{enumerate}

\subsection*{Working Papers}

\begin{enumerate}

\item \href{https://ts-robinson.com/files/tsr_ballot_initiatives_confirmation.pdf}{``Direct democracy in representative systems: Understanding breakdowns in responsiveness through ballot initiative success."} 
% \begin{adjustwidth}{0.5cm}{0.5cm}
%  \textit{Why do ballot initiatives succeed in representative democratic systems? I test a series of theories using precinct-level voter returns, individual-level estimates of donors' ideology, and qualitative evidence from state legislators. I argue that initiatives typically succeed when issues have not been incorporated into the political mainstream.}  \\
%  \end{adjustwidth}

\item \href{https://ts-robinson.com/files/campaign_finance_tsr_confirmation.pdf}{``Knowledge or ignorance? Assessing the causal effects of campaign finance disclosure on ballot decisions."}
% \begin{adjustwidth}{0.5cm}{0.5cm}
%  \textit{I use a series of conjoint survey experiments on US respondents to show that the effects of disclosure are negligible in the presence of partisan and ideological cues. This is the first paper to test the effects of disclosure across direct democratic and representative institutions.}
%  \end{adjustwidth}

\item \href{https://ts-robinson.com/files/standard_errors_tsr.pdf}{``When should we cluster experimental standard errors?" }
% \begin{adjustwidth}{0.5cm}{0.5cm}
%  \textit{Despite their ubiquity in analyses with group-constant variables, the rationale for using standard errors in experimental contexts remains underdeveloped. I show when and why experimentalists should use cluster-robust standard errors, and replicate existing studies to demonstrate the importance of correctly doing so.} \\
%  \end{adjustwidth} 

%\\
\end{enumerate}


\subsection*{Work in Progress}

\begin{enumerate}
\item ``Gender Differences and Development? A Non-Parametric Bayesian Perspective.'' With Raymond Duch and Luke Keele. 
% \begin{adjustwidth}{0.5cm}{0.5cm}
% \textit{We use machine learning to re-examine differences in economic behaviour between men and women from a global survey of preferences. We argue that in contexts where preferences are likely to be highly heterogeneous, researchers need to pay close attention to the functional form of their models.} \\
% \end{adjustwidth}

\item Power analysis for experimental designs.  With Raymond Duch.
% \begin{adjustwidth}{0.5cm}{0.5cm}
% \textit{We provide a non-parametric strategy for estimating the necessary sample sizes when conducting experiments.}
% \end{adjustwidth}
\end{enumerate}

 \subsection*{Book Reviews}

  Robinson, T. S. (2018), ``Brexit: Why Britain Voted to Leave the European Union by Harold D. Clarke, Matthew Goodwin, and Paul Whiteley."\textit{ Political Science Quarterly, 133(3): 564-565.}

 \section*{Presentations \& Conferences}

 \begin{tabular}{lll}
 	 2020 & EPSA & Panel \\
     2020 & PolMeth Europe & Poster \\
     2019 & ``Crashing out of Politics" inequality workshop & Convenor $+$ Panel \\
     2019 & DPIR Politics Research Colloquium & Presenter \\
     2019 & CESS Social Media Workshop, Nuffield College & Panel \\
     2019 & EPSA  & Panel $+$ Panel Chair \\
     2019 & Oxford-LSE Politics Graduate Student Conference & Panel \\
     2019 & American Politics Graduate Seminar, Rothermere American Institute & Presenter \\
     2018 & ``US Politics after the Midterms", St Anne's College & Talk \\
     2018 & DPhil Politics/IR Seminar, DPIR, University of Oxford & Panel \\
     2018 & King's College London JMCE Research Workshop & Panel \\
     2018 & American Politics Group (PSA) conference & Panel \\
     2017 & American Politics Group (PSA) conference & Panel \\
     2017 & Departmental Seminar, Blavatnik School of Government & Presenter \\
     2016 & Midwest Political Science Association (MPSA) & Poster \\
     2015 & American Politics Graduate Seminar, Rothermere American Institute & Presenter
 \end{tabular}

 \section*{Scholarships, Grants and Awards}

 \begin{tabular}{lll}
     2019 & Incubator Fund (to host conference, £600) & ESRC Grand Union DTP \\
     2019 & Academic Programme Fund (to host conference, £1000) & Rothermere American Institute \\
     2019 & Research Training Support Grant (Experiment, £1600) & ESRC Grand Union DTP \\
     2018 & Quantitative Graduate Methods Scholarship & Q-Step Centre, Oxford \\
     2018 & Mr Michell - Wadham RCUK Studentship & ESRC/Wadham College\\
     2017 & Richard E. Neustadt Essay Prize & American Politics Group (PSA) \\
     2016 & St Anne's Centenary Bursary & St Anne's College, Oxford \\
     2016 & Travel Grant (MPSA Conference, £400) & St Anne's College, Oxford \\
     2015 & Danson Foundation Mentoring Programme & St Anne's College, Oxford \\
     2015 & Travel Grant (Fieldwork, £1000) & Rothermere American Institute \\
     2014 & Scholarship of the University of Oxford (renewed 2015) & St Anne's College

 \end{tabular}

 \section*{Teaching}
 \begin{tabular}{p{0.18\textwidth}|p{0.8\textwidth}}
     \textbf{Undergraduate} &  Q-Step 1 quantitative lab sessions, \textit{DPIR} \\
      & Practice of Politics (Prelims), \textit{St Anne's College}\\
      & Comparative Government (205), \textit{St Anne's College} \\
      & \\
      \textbf{Graduate} &  Introduction to Statistics, \textit{DPIR} \\
      & \\
      \textbf{Methods} & Machine Learning, \textit{Oxford Spring School in Advanced Research Methods 2020} \\
      & R for Reproducible Research, \textit{Oxford Spring School in Advanced Research Methods 2019} \\
      & Heterogeneity and Machine Learning Workshop, \textit{CESS/Essex Summer School} \\
      & Introduction to R, \textit{CESS/Essex Summer School}
 \end{tabular}

  \section*{Research Assistance}
  \begin{tabular}{ll}
    2016 - Present & \textbf{Centre for Experimental Social Sciences}, Nuffield College \\
    2018 & \textbf{Professor Andrew Eggers}, Nuffield College \\
    2017-18 & \textbf{Building Integrity Programme}, Blavatnik School of Government \\
 \end{tabular}

 \section*{Additional Courses}
 \begin{tabular}{ll}
     2018 & High-Performance Computing in Economics and the Social Sciences, Oxford\\
          & \textit{Taught by Prof. Jesús Fernández-Villaverde (UPenn)} \\
     2017 & Tutorial teacher training, DPIR Oxford\\
\end{tabular}

 \section*{Technical Skills}
 \begin{tabular}{ll}
     \textbf{Coding} & R, Julia, Python, Bash $|$ Working knowledge of MATLAB and VBA \\
     \textbf{Typesetting} & Latex, Markdown \\
     \textbf{Applications} & Stata, Qualtrics, Git(hub), Amazon AWS, Adobe Suite, Microsoft Office
 \end{tabular}
\end{document}
