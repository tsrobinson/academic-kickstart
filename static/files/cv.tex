\documentclass[11pt, a4paper]{article}
\usepackage[utf8]{inputenc}
\usepackage[a4paper, left = 1in, right = 1in, top = 0.7in, bottom = 0.7in]{geometry}
\usepackage{letltxmacro}
\usepackage{hyperref}

\setlength{\parindent}{0em}

\begin{document}
 \begin{center}
    \textsc{\huge Thomas Robinson} \\
    School of Government and International Affairs, Durham University \\
    +44 (0)7908 231 892 $|$ thomas.robinson@durham.ac.uk $|$ \href{https://ts-robinson.com}{ts-robinson.com}
 \end{center}

\section*{Academic Positions}
\begin{tabular}{ll}
    August 2020 & \textbf{Assistant Professor in Quantitative Comparative Politics} \\
    & School of Government and International Affairs \\
    & University of Durham, UK \\

 \end{tabular}
\vspace{-1em}
 \section*{Education}
 \begin{tabular}{ll}
    2020 & \textbf{University of Oxford} $|$ DPhil in Politics \\
    	 & \textit{Thesis Title:} Three Essays on Measuring Political Behaviour \\
    	 & \textit{Supervisor:} Prof. Andy Eggers\\
    	 & \textit{Examiners:} Prof. Ben Ansell (Oxford) and Prof. James Snyder (Harvard) \\
    	 & Passed without corrections. \\
    2018  & \textbf{University of Oxford} $|$ MPhil in Politics (Comparative Government) \\
    2016 & \textbf{University of Oxford} $|$ BA in Philosophy, Politics and Economics
 \end{tabular}

 \section*{Research}

 % \subsection*{Interests}
 % Ideology, representation and democracy $|$ Direct democratic policymaking $|$ Experimental and computational methodology.

 \subsection*{Peer-Reviewed Articles}

 \begin{enumerate}

 \item \href{https://ts-robinson.com/publication/gulzar-how-campaigns-respond-2020/gulzar-how-campaigns-respond-2020.pdf}{``How Campaigns Respond to Ballot Design? Evidence from Ballot Order Lotteries in Colombia.''} With Nelson Ruiz and Saad Gulzar. \textit{Journal of Politics}, (Accepted)

 \item \href{https://doi.org/10.1017/pan.2020.49}{``Applying the MIDAS Touch: How to Handle Missing Values in Large and Complex Data."} With Ranjit Lall. \textit{Political Analysis}, First View: 1-18.

 \item \href{https://doi.org/10.1017/psrm.2020.33}{``Nativist Policy: the comparative effects of Trumpian politics on migration decisions."} 2020. With Raymond Duch, Raymond, Denise Laroze, and Constantin Reinprecht. \textit{Political Science Research and Methods}, First View: 1-17.

 \item \href{https://www.cambridge.org/core/journals/political-analysis/article/multimodes-for-detecting-experimental-measurement-error/37514FC46CF29C7B345DB9881E252150/share/7b059037b0da9182a33316d7f87b2de81b619592}{``Multi-modes for Detecting Experimental Measurement Error."} 2020. With Raymond Duch, Denise Laroze, and Pablo Beramendi. \textit{Political Analysis}, 28(2): 263-283.

 \item \href{https://doi.org/10.1111/ssqu.12584}{``Where Will the British Go? And Why?"} 2019. With Raymond Duch, Denise Laroze, and Constantin Reinprecht. \textit{Social Science Quarterly}, 100(2): 480-493. 

\end{enumerate}


\subsection*{Under Review}

\begin{enumerate}

  \item \href{https://doi.org/10.1101/2021.01.31.21250866}{``Who should be first in line for the COVID-19 vaccine? Surveys in 13 countries of the public’s preferences for prioritisation."} With Raymond Duch et al (part of the international collaborative project \href{https://oxford-candour.com/}{CANDOUR}). 

  \item ``Efficient Multiple Imputation for Diverse Data in Python and R: \textbf{MIDASpy} and \textbf{rMIDAS}". With Ranjit Lall.
% \begin{adjustwidth}{0.5cm}{0.5cm}
%  \textit{This paper introduces \textbf{midas}, a Python class for multiply imputing missing values based on neural network methods that is particularly well suited to large and complex data.} \\
%  \end{adjustwidth}

\end{enumerate}

\subsection*{Working Papers}

\begin{enumerate}

\item \href{https://ts-robinson.com/publication/robinson-whenvotersrespond-2020/robinson-whenvotersrespond-2020.pdf}{``When do voters respond to campaign finance disclosure? Evidence from multiple election types"}
% \begin{adjustwidth}{0.5cm}{0.5cm}
%  \textit{I use a series of conjoint survey experiments on US respondents to show that the effects of disclosure are negligible in the presence of partisan and ideological cues. This is the first paper to test the effects of disclosure across direct democratic and representative institutions.}
%  \end{adjustwidth}

\item \href{https://ts-robinson.com/publication/robinson-directdemocracyrepresentative-2020/robinson-directdemocracyrepresentative-2020.pdf}{``Direct democracy in representative systems: Understanding breakdowns in responsiveness through ballot initiative success."}
% \begin{adjustwidth}{0.5cm}{0.5cm}
%  \textit{Why do ballot initiatives succeed in representative democratic systems? I test a series of theories using precinct-level voter returns, individual-level estimates of donors' ideology, and qualitative evidence from state legislators. I argue that initiatives typically succeed when issues have not been incorporated into the political mainstream.}  \\
%  \end{adjustwidth}


\item \href{https://ts-robinson.com/publication/robinson-whenshouldwe-2020/robinson-whenshouldwe-2020.pdf}{``When should we cluster experimental standard errors?" }
% \begin{adjustwidth}{0.5cm}{0.5cm}
%  \textit{Despite their ubiquity in analyses with group-constant variables, the rationale for using standard errors in experimental contexts remains underdeveloped. I show when and why experimentalists should use cluster-robust standard errors, and replicate existing studies to demonstrate the importance of correctly doing so.} \\
%  \end{adjustwidth} 

%\\
\end{enumerate}

\subsection*{Work in Progress}

\begin{enumerate}

\item ``Mind and machine: rooting out corrupt politicians'' With Nelson Ruiz and Ezequiel Gonzalez-Ocantos.

\item ``Heterogeneity analysis of marginal effects in conjoint experiments".  With Raymond Duch and Philip Clarke.
% \begin{adjustwidth}{0.5cm}{0.5cm}
% \textit{We provide a non-parametric strategy for estimating the necessary sample sizes when conducting experiments.}
% \end{adjustwidth}

\item ``Economic Beliefs and the Local Coronavirus Pandemic.'' With Raymond Duch and Peiran Jiao.

% \item ``Gender Differences and Development? A Non-Parametric Bayesian Perspective.'' With Raymond Duch and Luke Keele. 
% \begin{adjustwidth}{0.5cm}{0.5cm}
% \textit{We use machine learning to re-examine differences in economic behaviour between men and women from a global survey of preferences. We argue that in contexts where preferences are likely to be highly heterogeneous, researchers need to pay close attention to the functional form of their models.} \\
% \end{adjustwidth}

\end{enumerate}


\subsection*{Open-Source Software/Resources}

\begin{enumerate}

  \item \href{https://CRAN.R-project.org/package=rMIDAS}{``rMIDAS: Multiple Imputation Using Denoising Autoencoders''}. With Ranjit Lall and Alex Stenlake. \textit{R package, published on CRAN.}

  \item \href{https://pypi.org/project/MIDASpy/}{``MIDASpy''}. With Ranjit Lall and Alex Stenlake. \textit{Python package, published on PyPi.}

  \item \href{https://bookdown.org/ts_robinson1994/10_fundamental_theorems_for_econometrics/vtDma6bZJ/}{``10 Fundamental Theorems for Econometrics"}. \textit{Online statistics resource, available on bookdown.org}

\end{enumerate}


 \subsection*{Book Reviews}

  Robinson, T. S. (2018), ``Brexit: Why Britain Voted to Leave the European Union by Harold D. Clarke, Matthew Goodwin, and Paul Whiteley."\textit{ Political Science Quarterly, 133(3): 564-565.}

\section*{Scholarships, Grants and Awards}

\subsection*{Successful Grant Applications}
\begin{tabular}{lp{2.7cm}ll}
   2020 & Co-Investigator & Archive and Longitudinal Project, \textit{Education Endowment Fund} \\
   & & \textsterling1.7 million FEC; \textsterling762,000 cost to funder \\
 \end{tabular}

\subsection*{Pending Grant Applications}
\begin{tabular}{lp{2.7cm}ll}
   2020 & PI & Small Project Grant, \textit{BA/Leverhulme Trust} \\
   & & \textsterling33,000 FEC; \textsterling10,000 cost to funder \\
 \end{tabular}

\subsection*{Other Received Funding, Scholarships and Awards}
 \begin{tabular}{llll}
     2019 & Incubator Fund (to host conference, £600) & ESRC Grand Union DTP \\
     2019 & Academic Programme Fund (to host conference, £1000) & Rothermere American Institute \\
     2019 & Research Training Support Grant (Experiment, £1600) & ESRC Grand Union DTP \\
     2018 & Quantitative Graduate Methods Scholarship & Q-Step Centre, Oxford \\
     2018 & Mr Michell - Wadham RCUK Studentship & ESRC/Wadham College\\
     2017 & Richard E. Neustadt Essay Prize & American Politics Group (PSA) \\
     2016 & St Anne's Centenary Bursary & St Anne's College, Oxford \\
     2016 & Travel Grant (MPSA Conference, £400) & St Anne's College, Oxford \\
     2015 & Danson Foundation Mentoring Programme & St Anne's College, Oxford \\
     2015 & Travel Grant (Fieldwork, £1000) & Rothermere American Institute \\
     2014 & Scholarship of the University of Oxford (renewed 2015) & St Anne's College

 \end{tabular}

 \section*{Teaching}

  \begin{tabular}{p{0.18\textwidth}|p{0.8\textwidth}}
     \textbf{Undergraduate} 
      & Researching Politics and International Relations, \textit{SGIA, Durham University}\\
      & Q-Step 1 quantitative lab sessions, \textit{DPIR, University of Oxford} \\
      & Practice of Politics (Prelims), \textit{St Anne's College, University of Oxford}\\
      & Comparative Government (205), \textit{St Anne's College} \\
      & \\
      \textbf{Graduate} 
      & Statistical Exploration and Reasoning, \textit{SGIA}, \textbf{Module Convenor} \\
      & Designing Political Inquiry, \textit{SGIA} \\
      &  Introduction to Statistics, \textit{DPIR} \\
      & \\
      \textbf{Methods Schools} & Machine Learning in the Social Sciences, \textit{Oxford Spring School in Advanced Research Methods 2021} \\
      & R for Reproducible Research, \textit{Oxford Spring School 2019} \\
      & Heterogeneity and Machine Learning Workshop, \textit{CESS-Essex Summer School 2019} \\
      & Introduction to R, \textit{CESS-Essex Summer School 2019} \\
      & \\
      \textbf{Public Sector} 
      & Advanced Course on Data Analytics for the Public Sector, \textit{Asian Productivity Organization} \\
 \end{tabular}

 \section*{Professional Activities}

 \subsection*{Accepted Presentations and Seminars}

 \begin{tabular}{lll}
     2021 &  Experimental Insights from Behav. Econ. on Covid-19, JHU \& LSE & Presenter \\
     2020 & ``Flipped Seminar" Series, CIPB, Durham University & Presenter \\
     2020 & Work-in-Progress Seminar Series, DPIR, University of Oxford & Presenter \\
     2020 & EPSA [cancelled due to COVID-19] & Panel \\
     2020 & PolMeth Europe [cancelled due to COVID-19] & Poster \\
     2019 & ``Crashing out of Politics" inequality workshop & Convenor $+$ Panel \\
     2019 & DPIR Politics Research Colloquium & Presenter \\
     2019 & CESS Social Media Workshop, Nuffield College & Panel \\
     2019 & EPSA  & Panel $+$ Panel Chair \\
     2019 & Oxford-LSE Politics Graduate Student Conference & Panel \\
     2019 & American Politics Graduate Seminar, Rothermere American Institute & Presenter \\
     2018 & ``US Politics after the Midterms", St Anne's College & Talk \\
     2018 & DPhil Politics/IR Seminar, DPIR, University of Oxford & Panel \\
     2018 & King's College London JMCE Research Workshop & Panel \\
     2018 & American Politics Group (PSA) conference & Panel \\
     2017 & American Politics Group (PSA) conference & Panel \\
     2017 & Departmental Seminar, Blavatnik School of Government & Presenter \\
     2016 & Midwest Political Science Association (MPSA) & Poster \\
     2015 & American Politics Graduate Seminar, Rothermere American Institute & Presenter
 \end{tabular}

 \subsection*{Service Positions}

 \begin{tabular}{lll}
   2020- & Deputy Director of Doctoral Training Centre & SGIA, Durham University \\
   2019-20 & Head of Online Experiments & CESS, Nuffield College \\
   
 \end{tabular}

 %  \section*{Research Assistance}
 %  \begin{tabular}{ll}
 %    2016 - 2020 & \textbf{Centre for Experimental Social Sciences}, Nuffield College \\
 %    2018,2020 & \textbf{Professor Andrew Eggers}, Nuffield College \\
 %    2017-18 & \textbf{Building Integrity Programme}, Blavatnik School of Government \\
 % \end{tabular}

%  \section*{Additional Courses}
%  \begin{tabular}{ll}
%      2018 & High-Performance Computing in Economics and the Social Sciences, Oxford\\
%           & \textit{Taught by Prof. Jesús Fernández-Villaverde (UPenn)} \\
%      2017 & Tutorial teacher training, DPIR Oxford\\
% \end{tabular}

 \section*{Technical Skills}
 \begin{tabular}{ll}
    \textbf{Coding languages} & R, Julia, Python, Bash $|$ Working knowledge of MATLAB and VBA \\
    \textbf{Systems} & Git(hub), Amazon AWS \\
    \textbf{Courses} & High-Performance Computing in Economics and the Social Sciences \\
    & \textit{Taught by Prof. Jesús Fernández-Villaverde (UPenn)} \\
     
 \end{tabular}
\end{document}
